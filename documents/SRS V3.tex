\documentclass[12pt]{article}
\usepackage{geometry}
\usepackage{multirow}
\usepackage{enumerate} 
\geometry{a4paper} 

\begin{document}

\title{Software Requirement Specification}
\author{}
\date{}
\maketitle

\section{Introduction}

\subsection{Project Identification}

\begin{table}[ht]
\begin{tabular}{|l|l|}
\hline 
\textbf{Project Title} & Visualization of sorting algorithms \\
\hline
\textbf{Software Name} & Donimo \\
\hline
\textbf{Crouse Name} & Software Engineering Group Project \\
\hline
\textbf{Crouse Identifier} & AE2GRP   \\
\hline
\textbf{Project Supervisor} & Heshan Du   \\ 
\hline
\textbf{Group Number} & 5     \\
\hline
\textbf{Group Members}&Zhefeng Zhou, Yangyu Gao, Muyi Jiang\\ &Jiaying Sun, Kan Liu, Zhe Ren\\
\hline
\textbf{Start Date} & September 14, 2016  \\ 
\hline
\textbf{End Date} & May 3, 2017   \\
\hline
\end{tabular}
\end{table}

\subsection{Purpose}

This Software Requirement Specification (SRS) identifies the requirements and specification for the software of this project. It explains the functional features of the animation together with design, interface details and other considerations.

\subsection{Scope}

The software is designed to provide visualization of sorting algorithms. The main purpose is to help users understand the principle and efficiency of sorting algorithms. Users should have basic understanding of what algorithm is, such as computer science students or those interested in sorting algorithm.

\subsection{Overview}

The rest of the SRS presents the specifications of the software in detail. Section 2 of the SRS is the overall descriptions of the software and its requirements. Section 3 outlines other related requirements of the software.  

\section{Overall Descriptions}

\subsection{Product Perspectives}

The software is an independent and totally self-contained product intended for use on the windows platform.

\subsection{Product Functions}

The list provides a description of main features of function. These features are divided to two categories: core features and optional features. Core features are essential to the operation of the application, and optional features are additional functionalities.

\subsubsection{Core features}


\begin{enumerate}
\item The software should provide users a guide of how to use core features.  

\item The software should allow users to select which sorting algorithm to be animated.

\item The software should visualize the processes of running sorting algorithm. For example, bars' movement represent the output of two input numbers's comparison when users choose bar chart to virsualize the sorting algorithms. 

\item The software should allow users to start, stop, slow down, speed up, restart or pick up a point of the animation.

\item The software should show the running soring algorithm’s source code to users.

\item The software should match specific line of running soring algorithm’s source code and it's explanation with animation.

\item The software should allow users to compare the efficiency of different sorting algorithms by showing the amination of them simultaneously in the interface.


\end{enumerate}

\subsubsection{Optional features}

\begin{enumerate}
\item The software should provide the voice service of algorithms’ explanation. 

\item The software should allow user to share this application to Social Network Services(SRS) website.

\item The software should allow user to customize the preferences of interface.   

\item The software should allow user to download the source code.

\end{enumerate}

\section{Specific Requirements}

\begin{enumerate}
\item Capacity\\The maximum number of sorting algorithms in comparison function is limited to 2. 

\item Software Language\\ All coding will be done in Java SE8 and CSS.  

\item Online User Documentation and Help System Requirements\\The software should allow user to download the source code to local address. All documentation will be made in accordance with requirements pertaining to open source software under the GNU General Purpose License. Additionally, on-line user documentation will be in the form of Java Platform, Standard Edition 8 API Specification.

\end{enumerate}

\end{document}
