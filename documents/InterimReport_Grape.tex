% ------------------------------------------------------------------------------
% LaTeX Template: Titlepage
% This is a title page template which be used for both articles and reports.
%
% Copyright: http://www.howtotex.com/
% Date: April 2011
% ------------------------------------------------------------------------------

% -------------------------------------------------------------------------------
% Preamble
% -------------------------------------------------------------------------------
\documentclass[paper=a4, fontsize=11pt,twoside]{scrartcl}		% Koma article

\usepackage[a4paper,pdftex]{geometry}										% A4paper margins
\setlength{\oddsidemargin}{5mm}												% Remove 'twosided' indentation
\setlength{\evensidemargin}{5mm}


\usepackage{fancyhdr} % Required for custom headers
\usepackage{lastpage} % Required to determine the last page for the footer
\usepackage{extramarks} % Required for headers and footers

\usepackage[english]{babel}
\usepackage[protrusion=true,expansion=true]{microtype}	
\usepackage{amsmath,amsfonts,amsthm,amssymb}
\usepackage{graphicx}
\usepackage[nottoc,numbib]{tocbibind} % Required for inlucde reference in TOC
\usepackage[pdfstartview=FitH,
CJKbookmarks=true,
bookmarksnumbered=true,
bookmarksopen=true,
colorlinks,
pdfborder=001,
linkcolor=black,
citecolor=blue,
]{hyperref} % Required for hyper link references

% ------------------------------------------------------------------------------
% Definitions (do not change this)
% ------------------------------------------------------------------------------
\newcommand{\HRule}[1]{\rule{\linewidth}{#1}} 	% Horizontal rule

\makeatletter							% Title
\def\printtitle{%						
    {\centering \@title\par}}
\makeatother									

\makeatletter							% Author
\def\printauthor{%					
    {\centering \large \@author}}				
\makeatother							

% ------------------------------------------------------------------------------
% Metadata (Change this)
% ------------------------------------------------------------------------------
\title{	\normalsize \textsc{Title page subtitle} 	% Subtitle of the document
		 	\\[2.0cm]													% 2cm spacing
			\HRule{0.5pt} \\										% Upper rule
			\LARGE \textbf{\uppercase{Interim Report}}	% Title
			\HRule{2pt} \\ [0.5cm]								% Lower rule + 0.5cm spacing
			\normalsize \today									% Todays date
		}

\author{
		Group 5\\	
		The University Of Nottingham Ningbo China\\	
        }
    
\begin{document}
% ------------------------------------------------------------------------------
% Maketitle
% ------------------------------------------------------------------------------
\thispagestyle{empty}				% Remove page numbering on this page
\printtitle									% Print the title data as defined above
  	\vfill
\printauthor								% Print the author data as defined above
\cleardoublepage

% ------------------------------------------------------------------------------
%  table of contents
% ------------------------------------------------------------------------------
\newpage
\tableofcontents
\thispagestyle{empty}	
\newpage
\cleardoublepage

% ------------------------------------------------------------------------------
% Begin document
% ------------------------------------------------------------------------------

\setcounter{page}{1}

% ------------------------------------------------------------------------------
% INTRODUCTION      Kan LIU
% ------------------------------------------------------------------------------
\section{Introduction}
\subsection{Description}
\subsection{Goal}
\subsection{Roles of member }


% ------------------------------------------------------------------------------
% Background Research   Kan LIU, Jiaying SUN, Zhe REN
% ------------------------------------------------------------------------------
\section{Background Research}
\subsection{Technical research}
% Ninja progess
For the main frame of the sorting algorithm animation, the interface is divided into 5 parts. Each of these 5 parts is not independent. 
\begin{enumerate}
	\item For platform, we choose PC to be our main operating platform. One side of the reason we choose PC is the limited personal programming ability. We don’t have experience in mobile application developing. Therefore, developing on other platforms may take us a lot of time to learn a new programming skill and the outcome may not be guaranteed either. Another reason is that although PC is not so convenient as mobile device, it has own advantage. User may not care the space that a small application occupies, but in mobile device, because of the limited storage, a small application may cause the mobile phone running slowly. For different operating system, we choose window rather than IOS and other system. According to a research from StatCounter Global Stats, from 2015 to 2016, the number of Windows users still constitute a high proportion of PC users. This situation happens not only in China, but also in the whole world. Therefore, it’s better for us to develop for the majority.
	\item For developing tools, we use eclipse to be our program editor and compiler because we learn a lot skills about eclipse in class and it’s also convenient for us to test our code using Junit. We choose Java scene builder, which can also be used in eclipse, as our GUI designing tools. This software have a visualization of interface designing. It is more convenient for us to drug the function part into a visual interface rather than just write abstract code. 
	\item The reason why we choose Java as our programming language is that we learn more about that language in class and get more familiar with it. Our supervisor can also give us more suggestion about Java programming. 
	
\end{enumerate}


% ------------------------------------------------------------------------------
% Requirement Specification
% ------------------------------------------------------------------------------
\section{Requirement Specification}
\clearpage

% ------------------------------------------------------------------------------
% Design      Muyi JIANG ,Yangyu GAO
% -----------------------------------------------------------------------------
\section{Design}
\subsection{Main Frame}
% Ninja progess
For the main frame of the sorting algorithm animation, the interface is divided into 5 parts. Each of these 5 parts is not independent. 
\begin{enumerate}
   \item The main part is the animation. This part include several data and the geometric figure generated by them. When program is running, they will move on the basis of algorithm codes. The place changing can easily explain how the selected algorithm works. 
   \item The second part, which is under the animation, is control function. It contains 5 functional buttons and they work together to control the animation of algorithm. The first button on the left-up corner of this part is the “play-stop” button. When the “stop” button is click, the animation will freeze at the present step. Then the “stop” button will become the “paly” button. The animation will not continue the rest step until the “play” button is clicked. 
   \item The controller on the right side of the “play-stop” button is progress line. User can drug the button on the line to control the progress of animation. The animation will stop at differential steps with differential positions. 
   \item The button at the lower left corner is the speed bar. The speed input box is on the right side. These two controllers is used to control the speed of animation. User can change the speed of figure moving through drugging the control line or inputting an integer level from 1 to 10. 
   \item The last button at the lower right corner in this part is shape chosen box. User can choose or change the shape of figures in this choose box. Several figures such as rectangle, triangle and circle are provided for users. This function will not influence the process of algorithm code running. 

\end{enumerate}


% ------------------------------------------------------------------------------
% Implementation     Zhe REN, Jiaying SUN, Yangyu GAO
% -----------------------------------------------------------------------------
\section{Implementation}
\clearpage

% ------------------------------------------------------------------------------
% Progress Report    Zhefeng ZHOU
% -----------------------------------------------------------------------------
\section{Progress Report}
This section mainly covers 3 parts, progress we made, problems we met and our time plan. The first subsection summarizes the progress we made so far, the second subsection discusses some problems encountered, including both technical and management issues, and the third subsection talks about the future time plan of the project. \\
Reference Example\cite{Debray:2000:CTC:349214.349233}

\subsection{Progress to date}
% FINGAL progess
As mentioned in the previous sections, a portion of this project is dedicated to research and investigation in order to elicit the requirements, best technologies for the job and conduct some feasibility studies on past existing or novel solutions that solve some part(s) or all of the problem. It is therefore important to emphasize the role these play and their sizable contribution to the progress made. The progress thus far is as follows:
\begin{enumerate}
	\item \textbf{Project Website: } The project website is using Jekyll and is free hosting in GitHub Pages, which link is \url{grapeUNNC.github.io}. The website provide basic introduction of the project and the team role.
	\item \textbf{Requirements Specification: }The Functional and Non-Functional requirements specifications were determined and enumerated. The elicitation of these specifications had an effect on the time frame of the project such that it necessarily had to be adjusted to account for further research and implementation components.
	\item \textbf{System Design: }Given the requirements of the system, it was necessary to formulate a design which would be followed in the implementation of the system. This also added to the direction of the project so that the Feasibility Study and Prototyping stages were better informed with respect to suitability.
	\item \textbf{Feasible Study: }An evaluation of current novel and existing technologies and systems is made in order to best determine the extent to which they solved the problem(s) outlined in the Specifications. A decision was also made on which of these would be used in order to progress with this project and solve the problems outlined therein.
	\item \textbf{Prototype: }Basic prototyping was done following the designs of the system to trial run some of the technologies chosen for parts of the project and evaluate the scope, direction and projections of the project.
\end{enumerate}
\subsection{Problems encountered}
% FINGAL problems
\subsection{Time plans for the next half}
The second half of the cycle will be composed largely of implementation, testing and debugging steps in order to realize the design.
% INCLUDE GANTT CHART HERE

\clearpage

% ------------------------------------------------------------------------------
% Appendices
% ------------------------------------------------------------------------------
\section{Appendices}
\clearpage

% ------------------------------------------------------------------------------
% Bibliography
% ------------------------------------------------------------------------------
\bibliographystyle{plain}
\bibliography{InterimReport_Grape}

% ------------------------------------------------------------------------------
% End document
% ------------------------------------------------------------------------------
\end{document}










